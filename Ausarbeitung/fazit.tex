Hier könnte dein Fazit stehen... :)\\
Die Monte-Carlo-Methode mit Metropolis-Algorithmus ist eine simple Möglichkeit, das Ising-Modell zu simulieren und führt zumindest im 2-dimensionalen schnell zu anschaulichen Ergebnissen. So erhalten wir eine kritische Temperatur $ T_c \approx 2.26 $ (analytisch: $\approx 2.2692$) Im 3-dimensionalen, wo es keine analytische Lösung gibt, ist die Rechenzeit deutlich erhöht, aber immernoch nicht zu extrem.\\
Außerdem lässt sich der Einfluss eines äußeren Magnetfeldes studieren, während man aber bereit sein muss, eine gewisse Anzahl an Rechenschritten am Anfang zu ignorieren, bis das System konvergiert ist. Des Weiteren kommt es noch nach der Konvergenz zu einem relativ breiten Rauschen um den erreichten Wert.\\
Die Untersuchung des Cluster-Update-Verfahrens durch den Einsatz des Wolff-Algorithmus hat gezeigt, dass dieser sowohl Vor- als auch Nachteile bietet.\\
Als Vorteil ist sicherlich die schnellere Konvergenz zu erwähnen, durch die massiv Rechenzeit eingespart werden könnte. Zudem wird der "critical-slowdown" den ein lokales Verfahren mit sich bringt verhindert.\\
Das Cluster-Update ist allerdings nicht brauchbar, wenn der exakte Verlauf der Magnetisierung untersucht werden soll, da man die Absolutwerte der Messergebnisse betrachten muss. Zudem ist die Einwirkung eines externen Magnetfeldes nicht sinnvoll untersuchbar.
 
