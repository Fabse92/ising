
Die Monte-Carlo-Methode mit Metropolis-Algorithmus ist auf Grund der Konvergenz der makroskopischen Größen eine simple Möglichkeit, das Ising-Modell zu simulieren und führt für zwei- und dreidimensionale Gitter schnell zu anschaulichen Ergebnissen, die den Zusammenhang zwischen diesen Größen darstellen.\\
Die Untersuchung der Magnetisierung in Abhängigkeit der Temperatur zeigte einen Phasenübergang zwischen Hoch- und Niedertemperaturphase auf. So erhielten wir für das 2D-Modell eine kritische Temperatur von $ T_c \approx 2.26 \pm 0,02$ (analytisch: $\approx 2.2692$) für das 3D-Modell lag diese bei $ T_c \approx 4.50 \pm 0,02$. Hier gibt es allerdings keinen analytischen Vergleichswert, jedoch liegt er im Bereich anderer Simulationen. Bei Einwirken eines externen Magnetfelds konnte die typischen Ausschmierungen beobachtet werden.\\
???Außerdem lässt sich der Einfluss eines äußeren Magnetfeldes studieren, während man aber bereit sein muss, eine gewisse Anzahl an Rechenschritten am Anfang zu ignorieren, bis das System konvergiert ist.??? Des Weiteren kommt es noch nach der Konvergenz zu einem relativ breiten Rauschen um den erreichten Wert.\\
Die Untersuchung des Cluster-Update-Verfahrens durch den Einsatz des Wolff-Algorithmus hat gezeigt, dass dieser sowohl Vor- als auch Nachteile bietet.\\
Als Vorteil ist sicherlich die schnellere Konvergenz zu erwähnen, durch die massiv Rechenzeit eingespart werden könnte. Zudem wird der "critical-slowdown" den ein lokales Verfahren mit sich bringt verhindert.\\
Das Cluster-Update ist allerdings nicht brauchbar, wenn der exakte Verlauf der Magnetisierung untersucht werden soll, da man die Absolutwerte der Messergebnisse betrachten muss. Zudem ist die Einwirkung eines externen Magnetfeldes nicht sinnvoll untersuchbar.
 
