
Die Monte-Carlo-Methode mit Metropolis-Algorithmus ist auf Grund der Konvergenz der makroskopischen Größen eine simple Möglichkeit, das Ising-Modell zu simulieren und führt für zwei- und dreidimensionale Gitter schnell zu anschaulichen Ergebnissen, die den Zusammenhang zwischen diesen Größen darstellen. Diese Konvergenz ist jedoch nur bedingt exakt, da ein relativ breites Rauschen um den erreichten Wert eintritt, welches vor allem im Bereich um die kritische Temperatur sehr stark wird. Dennoch bietet schon das simple Verfahren eine gute Näherung für die Darstellung der Zusammenhänge.


Die Untersuchung der Magnetisierung in Abhängigkeit der Temperatur zeigte einen Phasenübergang zwischen Hoch- und Niedertemperaturphase auf. So erhielten wir für das 2D-Modell eine kritische Temperatur von $ T_c \approx 2.26 \pm 0,02$ (analytisch: $\approx 2.2692$) für das 3D-Modell lag diese bei $ T_c \approx 4.50 \pm 0,02$. Hier gibt es allerdings keinen analytischen Vergleichswert, jedoch liegt er im Bereich anderer Simulationen. Bei Einwirken eines externen Magnetfelds konnte die typischen Ausschmierung der Kurve beobachtet werden.


Weiterhin lässt sich die Magnetisierung in Abhängigkeit eines äußeren Magnetfeldes studieren, was Aufschluss über das magnetische Schaltverhalten des System liefert. Dabei konnte die für Ferromagneten typischen Hystereseschaltungen bei geringen Temperaturen beobachtet werden, während für höhere Temperaturen ein einfaches Schaltverhalten zu erkennen war.


Die Untersuchung des Cluster-Update-Verfahrens durch den Einsatz des Wolff-Algorithmus hat gezeigt, dass dieser sowohl Vor- als auch Nachteile bietet.\\
Als Vorteil ist sicherlich die schnellere Konvergenz zu erwähnen, durch die massiv Rechenzeit eingespart werden könnte. Zudem wird der "critical-slowdown" den ein lokales Verfahren mit sich bringt verhindert.\\
Das Cluster-Update ist allerdings nicht brauchbar, wenn der exakte Verlauf der Magnetisierung untersucht werden soll, da man die Absolutwerte der Messergebnisse betrachten muss. Zudem ist die Einwirkung eines externen Magnetfeldes nicht sinnvoll untersuchbar.
 
