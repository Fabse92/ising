
Durch die Konvergenz der Magnetisierung über eine Monte Carlo Simulation kann nun das Verhalten der Magnetisierung in Abhängigkeit von äußeren Parametern wie Temperatur und Magnetfeld untersucht werden. Hierbei ist zu beachten, dass für die Kopplungskonstante und den Boltzmannfaktor im Folgenden gilt: $J=k_{b}=1$. Insbesondere folgt daraus, dass hier nur Ferromagneten betrachtet werden. Für Antiferromagneten müssten alle Simulationen mit $J=-1$ durchgeführt werden.


\subsection{Phasenübergang zwischen Hoch- und Niedertemperaturphase}
\label{auswT}

Um die Abhängigkeit der Magnetisierung von der Temperatur zu bestimmen, wird jeweils eine Monte Carlo Simulation pro Temperatur für ein konstantes Magnetfeld durchgeführt. Dabei kennzeichnet die Temperatur, an der die Magnetisierung auf Null zurückgeht einen kritischen Punkt, an dem ein Phasenübergang stattfindet.


\paragraph*{2D-Modell}

\

\

Als erstes wird das 2D-Ising Modell betrachtet. Als Grundlage dient ein 50x50 Gitter. Es wird je Simulation mit einem neuen Gitter gestartet und 1000 Monte Carlo Schritte ausgeführt. 


Zunächst wird die Temperaturabhängigkeit ohne äußeres Feld ($B=0$) untersucht. Durch die analytische Lösung ist die kritische Temperatur ($T_{c}\approx 2,269$), welche den Übergang zwischen Hoch- und Niedertemperaturphase bestimmt, bekannt, sodass eine Simulation von $T=1,5$ bis $T=3,5$ mit einem Schritt von $\Delta T = 0.005$ den interessanten Bereich beinhaltet. Hierbei werden zunächst die unterschiedlichen Auswirkungen der Startkonfigurationen betrachtet (Abb. \ref{mp2d0modes}).
\begin{figure}[H]
	\centering
	\subfigure[Start aus r-Konfiguration]{
		\includegraphics[width=0.45\textwidth]{../Graph_Export/MP2D/m(T)_B=0_rMode_MP2D_50_Plot.jpg}
}	
	\subfigure[Start p- und n-Konfigurationen]{
		\includegraphics[width=0.45\textwidth]{../Graph_Export/MP2D/m(T)_B=0_pnModes_MP2D_50_Plot.jpg}
}		
	\caption{Temperaturabhängigkeit der Magnetisierung im 2D Ising Modells via Metropolisalgorithmus ohne äußeres Feld}
	\label{mp2d0modes}
\end{figure}
Es ist erkennbar, dass sich die Wahl der Startkonfiguration nicht auf den grundsätzlichen Verlauf der Kurve auswirkt. In allen Fällen konvergiert die Simulation gegen den Wert $|m_0|$. Je niedrigere die Temperatur desto höher ist der Wert $|m_0|$ desto stärker ist also das Gitter magnetisiert, bzw. sind die Spins parallel ausgerichtet. Ab einer gewissen kritischen Temperatur ist jedoch keine Magnetisierung des Gitters zu erkennen. Es sollten allerdings 2 Aspekte beachtet werden. Erstens erkennt man beim Start in einer r-Konfiguration einige Ausreißer im Bereich $T<T_{c}$, in denen einige Simulationen offensichtlich nicht konvergierten. Zweitens führt eine p-Startkonfiguration trivialerweise zu einer Konvergenz zum positiven Wert $m_{0}$ und umgekehrt. Daraus folgt, dass für diesen Bereich zwei Minima der freien Energie bei $\pm m_{0}$ existieren, welche bei Start in zufälliger Konfiguration gleich wahrscheinlich erreicht werden können, bei geordneter Startkonfiguration jedoch nur näheres erreicht wird. Daraus ergibt phänomenologisch ein Zusammenhang zwischen der freien Energie $F$ und der Magnetisierung $m$, wie in Abb. \ref{fmagT} dargestellt. Gleich bleibt jedoch das große Rauschen um $T_{c}$, welches zu einer großen Ungenauigkeit bei der Bestimmung der Curie-Temperatur führt, jedoch liegt sie mit einer Abschätzung von $T_{c}\approx 2,26\pm 0,02$ absolut im erwarteten Bereich. Hier ist eine Unstetigkeit in der ersten Ableitung zu erkennen, welche den Phasenübergang kennzeichnet. Auch ein kleineres Rauschen für $T>T_{c}$ mit einer Bandbreite von $\pm 0,2 = 0,4$ ist in allen Simulationen vorhanden. Um dieses Rauschen zu minimieren könnte man mit größeren Gittern arbeiten, diese erzeugen jedoch auch mehr Ausreißer im Bereich $T<T_{c}$.\\
Für eine optimale Auflösung empfiehlt sich für Metropolis Algorithmen ohne äußeres Feld folglich zwei Simulationen, je eine aus positiver und negativer Startkonfiguration.
\begin{figure}[H]
	\centering
	\includegraphics[width=0.4\textwidth]{../Graph_Export/F(mag)_T.jpg}
	\caption{schematischer Verlauf der freien Energie in Abhängigkeit der Magnetisierung bei verschiedenen Temperaturen}
	\label{fmagT}
\end{figure}
Nun soll die Auswirkung eines äußeren Feldes auf die Temperaturabhängigkeit betrachtet werden. Dazu wird obige Simulation für die Felder $B=\pm 0,1$ im Bereich $T=1,5$ bis $T=5,0$ und $B=\pm 0,4$ im Bereich $T=1,5$ bis $T=7,0$ durchgeführt. Dabei wird die Schrittweite auf $\Delta T = 0.01$ bzw. $\Delta T = 0.02$ erhöht (Abb. \ref{mp2db}). Für Simulationen mit äußerem Feld kann bedenkenlos aus zufälliger Konfiguration gestartet werden, da das angelegte Feld Ausreißer für $T<T_{c}$ verhindert und das Vorzeichen des angenommen Werts bestimmt. Bei geordneter Startkonfiguration ist die zweite Bedingung nicht zwingend erfüllt, wie Kapitel \ref{auswB} zeigen wird.


Während ohne äußeres Feld die Magnetisierung an $T_{c}$ steil abfällt und (bis auf ein Rauschen) auf Null zurückgeht, treten bei angelegtem Feld Ausschmierungen auf, sodass sich die Kurve asymptotisch der Null nähert und man weiterhin zwischen negativem und positivem Arm unterscheiden kann, welche dan jeweils mit etwa $0,2$ zum Grundrauschen beitragen. Auch nimmt die Größe der Ausschmierung bei stärkeren Feldern weiter zu.
\begin{figure}[H]
	\centering
	\includegraphics[width=0.85\textwidth]{../Graph_Export/MP2D/m(T)_MP2D_50_Plot.jpg}	
	\caption{Temperaturabhängigkeit der Magnetisierung im 2D Ising Modells via Metropolisalgorithmus für verschiedene äußere Felder}
	\label{mp2db}
\end{figure}


\paragraph*{3DModell}

\

\

Nun wird die Temperatur im 3D-IsingModell untersucht. Um in vertretbarem Rechenaufwand zu bleiben, wird ein 20x20x20 Gitter benutzt. Jede Simulation startet mit einem neuen Gitter und führt wiederum 1000 Schritte aus.


Als erstes wird wieder die Auswirkung der Startkonfiguration auf Simulationen ohne äußeres Feld untersucht (Abb. \ref{mp3d0modes}). Die Simulationen laufen jeweils von $T=2,5$ bis $T=6,0$ mit einer Schrittweite von $\Delta T= 0,01$.
\begin{figure}[H]
	\centering
	\subfigure[Start aus Zufallskonfiguration]{
		\includegraphics[width=0.45\textwidth]{../Graph_Export/MP3D/m(T)_B=0_rMode_MP3D_Plot.jpg}
}	
	\subfigure[Start aus geordneten Konfigurationen]{
		\includegraphics[width=0.45\textwidth]{../Graph_Export/MP3D/m(T)_B=0_pnModes_MP3D_Plot.jpg}
}		
	\caption{Temperaturabhängigkeit der Magnetisierung im 3D Ising Modells via Metropolisalgorithmus ohne äußeres Feld}
	\label{mp3d0modes}
\end{figure}
Zunächst ist erkennbar, dass der grundsätzliche Verlauf mit dem der 2D-Simulation übereinstimmt. Jedoch fällt auf, dass deutlich weniger Ausreißer im Bereich $T<T_{c}$ existieren und auch das Grundrauschen hat deutlich abgenommen. Es liegt nur noch bei etwa $\pm 0,1 = 0,2$. Ersteres liegt vermutlich an der kompakteren Struktur, also der höheren Anzahl an direkten Nachbarn. Zweiteres kommt durch die größere Anzahl an Spins (Gitterpunkten) zustande. Im 3D-Modell ist bei geeigneten Werten die Startkonfiguration auch für Simulationen ohne äußeres Feld zweitrangig. In beiden Fällen kann die Curie-Temperatur mit $T_{c}\approx 4,5\pm 0,02$ abgeschätzt werden.


Auch bei der Untersuchung der Einwirkung verschiedener äußerer Felder treten die gleichen Effekte analog zum 2D-Modell auf. Für die Simulation von $B=\pm 0,1$ wurden Temperaturen von $T=2,5$ bis $T=9,0$ mit Schrittweite $\Delta T= 0,02$ und für $B=\pm 0,4$ Temperaturen von $T=2,5$ bis $T=12,0$ mit Schrittweite $\Delta T= 0,025$ betrachtet (Abb. \ref{mp3db}).
 \begin{figure}[H]
	\centering
	\includegraphics[width=0.85\textwidth]{../Graph_Export/MP3D/m(T)_MP3D_Plot.jpg}	
	\caption{Temperaturabhängigkeit der Magnetisierung im 3D Ising Modells via Metropolisalgorithmus für verschiedene äußere Felder}
	\label{mp3db}
\end{figure}


\subsection{Schaltverhalten im Magnetfeld}
\label{auswB}

Als zweites soll die Abhängigkeit der Magnetisierung vom äußeren Feld betrachtet werden. Dies soll Aufschluss über das Schaltverhalten der Magnetisierung für verschiedene Temperaturen liefern. Dazu wird zu Beginn einmal ein Gitter erstellt auf welchem alle Simulationen der verschiedenen Feldstärken hintereinander ausgeführt werden, sodass ein tatsächlicher Sweep des externen Magnetfelds auf einem Gitter simuliert wird. Um das komplette Schaltverhalten zu beobachten, wird der Sweep in beiden Richtungen ausgeführt. Die Ergebnisse für das 2D- und 3D-Modell zeigt Abbildung \ref{mpbsweep}. Hierbei wurde das externe Feld jeweils zwischen $B=\pm1$ bzw $B=\pm1,5$ mit einer Schrittweite von $\Delta B=0,01$ variiert. Je Schritt des externen Felds wurde eine Simulation mit 1000 Monte Carlo Schritten ausgeführt. Für die 2D-Simulationen wurde erneut ein 50x50 Gitter und für die 3D-Simulationen ein 20x20x20 Gitter verwendet.
\begin{figure}[H]
	\centering
	\subfigure[2D Modell]{
		\includegraphics[width=0.45\textwidth]{../Graph_Export/MP2D/m(B)_MP2D_Plot.jpg}
}	
	\subfigure[3D Modell]{
		\includegraphics[width=0.45\textwidth]{../Graph_Export/MP3D/m(B)_MP3D_Plot.jpg}
}		
	\caption{Abhängigkeit der Magnetisierung von einem externen Magnetfeld, Simulation via Metropolisalgorithmus auf einem Gitter}
	\label{mpbsweep}
\end{figure}
In beiden Fällen ist zu erkennen, dass sich für $T<T_{c}$ eine Hysterese herausbildet, welche immer schmaler wird, je näher sich $T$ $T_{c}$ annähert. Zudem ist zu beobachten, dass für kleine Temperaturen das Gitter komplett schaltet, währen um $T_{c}$ herum die Magnetiesung vor dem Schalten langsam kolabiert. Dieses Schaltverhalten lässt sich anhand der zwei Minima der freien Energie an $\pm m_{0}$ erklären. Diese liegen je nach Temperatur mehr oder weniger tief, für $B=0$ jedoch auf selber Höhe (Gleichgewicht). Im Falle eines externen Feldes verschiebt sich das Gleichgewicht zu Gunsten des angelegten Feldes. Ein Umschalten findet folglich erst statt, wenn aus dem ungünstigen Extremum ein Wendepunkt geworden ist, was je nach Temperatur bei unterschiedlichen $|B|>0$ der Fall ist (Abb. \ref{fmagB}). An $T_{c}$ selbst ist erstmals keine Hysterese mehr zu erkennen, da ab hier nur noch ein einziges Minimum bei $m=0$ besteht (Abb. \ref{fmagT}). Für $T>T_{c}$ ist ein kontinuierlicher Übergang der Magnetisierung zu beobachten, welcher durch den Ursprung verläuft. Dies deckt sich mit den Ergebnissen der Simulationen zur Temperaturabhängigkeit Kapitel \ref{auswT}. Für höhere Temperaturen ist entsprechend ein stärkeres Feld nötig um eine komplette Magnetisierung zu erreichen. 
\begin{figure}[H]
	\centering
	\includegraphics[width=0.4\textwidth]{../Graph_Export/F(mag)_B.jpg}
	\caption{schematischer Verlauf der freien Energie in Abhängigkeit der Magnetisierung bei verschiedenen externen Feldern für eine Temperatur $T<T_c$}
	\label{fmagB}
\end{figure}

