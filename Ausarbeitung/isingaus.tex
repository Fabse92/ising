\documentclass[12pt,a4paper]{article}
\usepackage[latin1]{inputenc}
\usepackage{amsmath}
\usepackage{amsfonts}
\usepackage{amssymb}

\title{Monte Carlo Simulation des Ising-Modells}
\date{WS 14/15}
\author{Jan Fabian Schmid, Robert Hartmann, Niek Andresen}

\begin{document}
\maketitle

\section*{Das Ising-Modell}
Das Ising-Modell beschreibt n�herungsweise den Ferromagnetismus in Festk�rpern. Hierbei wird ein Kristall als �quidistantes Gitter gesehen, an dessen Gitterpunkten sich einzelne Spins befinden, welche ein magnetisches Moment haben. In der N�herung ver�ndern sich die Positionen der Spins nicht. Nimmt man zus�tzlich an, dass nur n�chste Nachbarn im Gitter sich gegenseitig beeinflussen und beschr�nkt man nun noch die Spins in ihre z-Richtung und l�sst sie nur zwei diskrete Werte annehmen ($\pm1$), ergibt sich der Hamiltonoperator:\\
\[
\hat{H}=-J\sum_{<i,j>} S_i S_j - B\sum_{i=1}^N S_i
\]
wobei $S_i \in \{-1,+1\}$. J ist die Kopplungskonstante zwischen benachbarten Spins, w�hrend die Summe nur �ber n�chste Nachbarn $<i,j>$ geht. B ist ein �u�eres Magnetfeld.\\
Somit liefern entgegengesetzt gerichtete Nachbarn einen positiven Energiebeitrag und gleichgerichtete einen negativen (f�r $J>0$).\\

\section*{Monte-Carlo-Simulation und Metropolis-Algorithmus}
Zur Simulation dieses Systems im Ising-Modell wird die Monte-Carlo-Methode verwendet. Hierbei werden mit Hilfe von Zufallszahlen m�gliche Konfigurationen des Systems erstellt. Um einen einfachen Mittelwert einer gesuchten Gr��e bilden zu k�nnen, werden Konfigurationen gem�� der Metropolis-Methode gew�hlt.\\
Die Wahrscheinlichkeitsverteilung ist die Boltzmann-Verteilung, sodass sich folgende Wahrscheinlichkeit zur Durchf�hrung eines Spin-Flips, der eine Energie�nderung von $\Delta E$ zur Folge h�tte, ergibt:
\[
P(S_k \rightarrow -S_k)=min\{ 1, e^{(-\frac{\Delta E}{k_B T})} \}
\]


\section*{Ziele}
Es soll das Verhalten des Gitters bei verschiedenen Temperaturen und �u�eren Magnetfeldst�rken sichtbar werden. Au�erdem wird die Konvergenz des Systems und damit der Gr��e der Gesamtmagnetisierung $m=\sum_i S_i$ sowie der Energie betrachtet.

\section*{Allgemeiner Programmablauf}
\end{document}